\section{Background}
\paragraph{} Significant technology advances have enabled planetary aircraft to be considered as viable science platforms.
Such systems fill a unique planetary science measurement gap, that of regional-scale, near-surface observation,
while providing a fresh perspective for potential discovery \cite{Braun2006}.

\paragraph{} Different approach have been made and among those, the most interesting at the time of conceiving our own design where:
\begin{description}
	\item[Plane lander] ARES mission from NASA, aims to control a powered plane during its guided descent on Mars. Final design has been decided\cite{Smith2004} and a half scale model test on high altitude Earth atmosphere.   \\
	\item[Inflatable wings plane] Similar to ARES missions have been conceived, to be more concrete, some researchers are trying to deploy inflatable wing structures in order to reduce weight and get more lift on the Mars atmosphere \cite{Simpson2005}.
	\item[Hybrid research drone] From German Aerospace Centre (DLR), a concept and approaches for hybrid Mars exploration UAV is outlined\cite{Singer2013}. 
\end{description}
