\section{Proposal}
\paragraph{} Mars atmosphere is so much thinner than Earth's one and also its principal gas is $CO_2$. These two facts compromise flying vehicles behaviour, that will require, new conceptual designs to operate as on Earth.

\paragraph{} Aiming to design a flying machine for Mars operations, we propose and hybrid aircraft that combines helicopter, plane and blimp performance into one model. The design and operation is basically inspired on the DLR hybrid Mars aircraft proposal\cite{Singer2013}, which outlined the main concepts to shape the idea.

\paragraph{} Our idea in common with DLR development\cite{Singer2013}, consist in building a blimp filled with a a gas lighter than $CO_2$. The lift force made by the Mars atmosphere to the blimp because its lower density, should be around the 80\%  of the total weight. \\

\noindent For VTOL, the blimp would had two wings with a motor on each end. The motors thrust will spin the blimp along its central axis producing a resulting upwards force for take off, like on helicopters.\\

\noindent An interesting fact about this idea, is that in case of thrust failure, the aircraft can descent like a parachute.\\

\noindent Finally, to moving from one point to another, the blimp could move its centre of gravity to the nose in order to glide to the surface, moving forward as it glides. Or, alternatively, the blimp could use its wing motors to propel the aircraft forward, sustained by wings lift along the atmosphere lift.\\

\noindent More details about its operation and appearance, can be found on DLR paper\cite{Singer2013} and  a feasibility study carried out with a \href{http://airshipworld.blogspot.com.es/2007/07/hybrid-airplane-new-hybrid-airship.html}{simplified experimental approach}.

