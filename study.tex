%
\typeout{}\typeout{If latex fails to find aiaa-tc, read the README file!}

\documentclass[]{aiaa-tc}% insert '[draft]' option to show overfull boxes
\usepackage{tabularx}
\usepackage{blindtext}


 \title{Conceptual design of a UAV for Mars}

 \author{
  Hector Anadón%
    \thanks{Aerospace engineering student, ESEIAAT.}
  \ and Pol Fontanes\thanksibid{1}\\
 }

 % Data used by 'handcarry' option if invoked
 %\AIAApapernumber{YEAR-NUMBER}
 %\AIAAconference{Conference Name, Date, and Location}
 %\AIAAcopyright{\AIAAcopyrightD{YEAR}}

 % Define commands to assure consistent treatment throughout document
 \newcommand{\eqnref}[1]{(\ref{#1})}
 \newcommand{\class}[1]{\texttt{#1}}
 \newcommand{\package}[1]{\texttt{#1}}
 \newcommand{\file}[1]{\texttt{#1}}
 \newcommand{\BibTeX}{\textsc{Bib}\TeX}

\begin{document}

\maketitle

\begin{abstract}
	First approach to Mars UAV conceptual design. Previous studies suggest that this kind of aerial machines could perform well on Mars. Background papers about this subject will be studied in order to present a UAV conceptual design.
\end{abstract}

\section*{Nomenclature}

\begin{tabbing}
  XXX \= \kill% this line sets tab stop
  $x$ \> Variable value vector \\
  $F$ \> Force, N \\
  $m$ \> Mass, kg \\
  $\Delta x$ \> Variable displacement vector \\
  $\alpha$ \> Acceleration, m/s\textsuperscript{2} \\
  $\rho$ \> Density, kg/m\textsuperscript{3} \\ [5pt]
  \textit{Subscript}\\
  $i$ \> Variable number \\
 \end{tabbing}

\section{Background}
\paragraph{} Significant technology advances have enabled planetary aircraft to be considered as viable science platforms.
Such systems fill a unique planetary science measurement gap, that of regional-scale, near-surface observation,
while providing a fresh perspective for potential discovery \cite{Braun2006}.

\paragraph{} Different approach have been made and among those, the most interesting at the time of conceiving our own design where:
\begin{description}
	\item[Plane lander] ARES mission from NASA, aims to control a powered plane during its guided descent on Mars. Final design has been decided\cite{Smith2004} and a half scale model test on high altitude Earth atmosphere.   \\
	\item[Inflatable wings plane] Similar to ARES missions have been conceived, to be more concrete, some researchers are trying to deploy inflatable wing structures in order to reduce weight and get more lift on the Mars atmosphere \cite{Simpson2005}.
	\item[Hybrid research drone] From German Aerospace Centre (DLR), a concept and approaches for hybrid Mars exploration UAV is outlined\cite{Singer2013}. 
\end{description}

\section{Proposal}
\paragraph{} Mars atmosphere is so much thinner than Earth's one and also its principal gas is $CO_2$. These two facts compromise flying vehicles behaviour, that will require, new conceptual designs to operate as on Earth.

\paragraph{} Aiming to design a flying machine for Mars operations, we propose and hybrid aircraft that combines helicopter, plane and blimp performance into one model. The design and operation is basically inspired on the DLR hybrid Mars aircraft proposal\cite{Singer2013}, which outlined the main concepts to shape the idea.

\paragraph{} Our idea in common with DLR development\cite{Singer2013}, consist in building a blimp filled with a a gas lighter than $CO_2$. The lift force made by the Mars atmosphere to the blimp because its lower density, should be around the 80\%  of the total weight. \\

\noindent For VTOL, the blimp would had two wings with a motor on each end. The motors thrust will spin the blimp along its central axis producing a resulting upwards force for take off, like on helicopters.\\

\noindent An interesting fact about this idea, is that in case of thrust failure, the aircraft can descent like a parachute.\\

\noindent Finally, to moving from one point to another, the blimp could move its centre of gravity to the nose in order to glide to the surface, moving forward as it glides. Or, alternatively, the blimp could use its wing motors to propel the aircraft forward, sustained by wings lift along the atmosphere lift.\\

\noindent More details about its operation and appearance, can be found on DLR paper\cite{Singer2013} and  a feasibility study carried out with a \href{http://airshipworld.blogspot.com.es/2007/07/hybrid-airplane-new-hybrid-airship.html}{simplified experimental approach}.


\section{UAV specifications}


\newpage
\bibliographystyle{unsrt}
\bibliography{UAV_sensors}

\end{document}